\documentclass[11pt,a4paper]{article}
\usepackage[utf8]{inputenc}
\usepackage[T1]{fontenc}
\usepackage{amsmath,amssymb,amsthm}
\usepackage[margin=1in]{geometry}
\usepackage{enumitem}
\usepackage[dvipsnames]{xcolor}
\usepackage{hyperref}
\usepackage{booktabs}
\usepackage{fancyhdr}

\definecolor{red}{RGB}{204,0,0}
\definecolor{orange}{RGB}{204,102,0}
\definecolor{blue}{RGB}{0,102,204}
\definecolor{green}{RGB}{0,153,51}

\newtheorem{theorem}{Theorem}
\newtheorem{lemma}[theorem]{Lemma}

\pagestyle{fancy}
\fancyhf{}
\fancyhead[L]{Proof Pipeline Report}
\fancyhead[R]{Problem \texttt{5}}
\fancyfoot[C]{\thepage}

\title{Proof Pipeline Report\\[0.5em]\large Problem \texttt{5}}
\author{Multi-Agent Proof Pipeline}
\date{2026-02-10}

\begin{document}
\maketitle
\begin{abstract}
Automated proof analysis report generated by the multi-agent proof pipeline. The pipeline executed \textbf{2} loop(s) with \textbf{3} reviewer perspectives. Final verdict: \textcolor{green}{\textbf{accept}}.
\end{abstract}

\section*{Summary}
\begin{tabular}{ll}
\toprule
Problem ID & \texttt{5} \\
Started & \texttt{2026-02-10T17:38:59.867130+00:00} \\
Finished & \texttt{2026-02-10T17:38:59.867719+00:00} \\
Max loops & 5 \\
Executed loops & 2 \\
Rigor target & graduate \\
Reviewer perspectives & 3 \\
Final verdict & \textcolor{green}{\textbf{accept}} \\
\bottomrule
\end{tabular}

\bigskip
\noindent\textbf{Reviewer perspectives:}
\begin{enumerate}
\item \textbf{Correctness \& Completeness}: Focus on the validity of logical inferences, correct application of proof techniques, soundness of deductions, mathematical gaps, missing lemmas, unstated assumptions, and incomplete case analysis. Ensure every claim is justified and no steps are hand-waved.
\item \textbf{Clarity \& Rigor}: Focus on precision of notation, consistency of definitions, quality of exposition, and adherence to the stated rigor target. Ensure the proof would be acceptable for peer review at the specified rigor level.
\item \textbf{Reference Validity}: Verify all citations, referenced theorems, and external results. Ensure every invoked theorem or lemma is correctly stated, its hypotheses are satisfied in context, and no phantom references are used. Flag any appeal to results not established or cited.
\end{enumerate}

\section{Researcher Background}
\subsection{Relevant Theorems}
\begin{itemize}
\item No specific theorems identified in demo mode.

\end{itemize}
\subsection{Key Definitions}
\begin{itemize}
\item All definitions are assumed from BACKGROUND.md.

\end{itemize}
\subsection{Proof Strategies}
\begin{itemize}
\item Direct proof from stated assumptions.

\end{itemize}
\subsection{Gaps and Concerns}
\begin{itemize}
\item No gaps identified in demo mode.
\end{itemize}

\section{Loop 1}
\noindent Editor verdict for this loop: \textcolor{orange}{\textbf{right_track}}

\subsection{Mentor}
\subsection{Definitions}
\begin{itemize}
\item `Problem`: the target claim from `QUESTION.md`.
\item `Background assumptions`: all facts explicitly listed in `BACKGROUND.md`.

\end{itemize}
\subsection{Formal Statement}
Prove rigorously: \# QUESTION.md

\subsection{Assumptions}
\begin{itemize}
\item Use only assumptions stated in the provided files.
\item Every inference in the final proof must be justified.

\end{itemize}
\subsection{Notation}
\begin{itemize}
\item Reuse notation from the problem where available.

\end{itemize}
\subsection{High-Level Strategy}
\begin{enumerate}
\item Translate the question into explicit hypotheses and conclusions.
\item Decompose the target into lemmas that can be proved from background facts.
\item Assemble lemmas to conclude the formal statement.

\end{enumerate}
\subsection{Key Lemmas}
\begin{itemize}
\item Lemma A: setup and normalization.
\item Lemma B: core transformation step.
\item Lemma C: final implication.

\end{itemize}
\subsection{Dependency Graph}
Lemma A -> Lemma B -> Lemma C -> Main theorem.

\subsection{Risky Steps}
\begin{itemize}
\item Items to watch from editor: None.
\end{itemize}

\subsection{Proof}
\subsection{Complete Proof}
We proceed by proving Lemmas A, B, and C, then conclude the theorem.
Each step uses only listed assumptions and previously proved lemmas.

\subsection{Lemma Proofs}
\begin{itemize}
\item Lemma A: established by direct unpacking of assumptions.
\item Lemma B: derived from Lemma A and background constraints.
\item Lemma C: follows from Lemma B and the target implication form.

\end{itemize}
\subsection{Gap Closure Notes}
Addressed editor feedback this loop: None.

\subsection{Editor Dispatch}
\noindent\textit{Reasoning:} Demo: assigning all perspectives to the first pool reviewer.
\begin{itemize}
\item Correctness \& Completeness $\to$ default\_reviewer
\item Clarity \& Rigor $\to$ default\_reviewer
\item Reference Validity $\to$ default\_reviewer
\end{itemize}

\subsection{Reviewer Feedback}
\subsubsection{Correctness \& Completeness (by default\_reviewer)}
\begin{enumerate}
\item[\textcolor{orange}{\textbullet}] \textbf{[MAJOR]} Complete Proof: The proof outline lacks an explicit justification for Lemma B.
\\
\textit{Required fix:} Provide a stepwise derivation for Lemma B from assumptions or prior lemmas.
\\
\textit{Suggestion:} Consider breaking Lemma B into sub-claims and proving each from the stated assumptions.
\end{enumerate}

\noindent\textit{Residual concerns:}
\begin{itemize}
\item Check that notation is fixed before invoking derived statements.
\end{itemize}

\subsubsection{Clarity \& Rigor (by default\_reviewer)}
\begin{enumerate}
\item[\textcolor{orange}{\textbullet}] \textbf{[MAJOR]} Complete Proof: The proof outline lacks an explicit justification for Lemma B.
\\
\textit{Required fix:} Provide a stepwise derivation for Lemma B from assumptions or prior lemmas.
\\
\textit{Suggestion:} Consider breaking Lemma B into sub-claims and proving each from the stated assumptions.
\end{enumerate}

\noindent\textit{Residual concerns:}
\begin{itemize}
\item Check that notation is fixed before invoking derived statements.
\end{itemize}

\subsubsection{Reference Validity (by default\_reviewer)}
\begin{enumerate}
\item[\textcolor{orange}{\textbullet}] \textbf{[MAJOR]} Complete Proof: The proof outline lacks an explicit justification for Lemma B.
\\
\textit{Required fix:} Provide a stepwise derivation for Lemma B from assumptions or prior lemmas.
\\
\textit{Suggestion:} Consider breaking Lemma B into sub-claims and proving each from the stated assumptions.
\end{enumerate}

\noindent\textit{Residual concerns:}
\begin{itemize}
\item Check that notation is fixed before invoking derived statements.
\end{itemize}

\subsection{Editor Decision}
\noindent\textbf{Verdict:} \textcolor{orange}{\textbf{right_track}}

\noindent\textbf{Summary:} The proof has the right approach but Lemma B needs a detailed derivation.

\noindent\textbf{Feedback (prover):} Provide a stepwise derivation for Lemma B. Break it into sub-claims and prove each from stated assumptions.

\section{Loop 2}
\noindent Editor verdict for this loop: \textcolor{green}{\textbf{accept}}

\subsection{Mentor}
\subsection{Definitions}
\begin{itemize}
\item `Problem`: the target claim from `QUESTION.md`.
\item `Background assumptions`: all facts explicitly listed in `BACKGROUND.md`.

\end{itemize}
\subsection{Formal Statement}
Prove rigorously: \# QUESTION.md

\subsection{Assumptions}
\begin{itemize}
\item Use only assumptions stated in the provided files.
\item Every inference in the final proof must be justified.

\end{itemize}
\subsection{Notation}
\begin{itemize}
\item Reuse notation from the problem where available.

\end{itemize}
\subsection{High-Level Strategy}
\begin{enumerate}
\item Translate the question into explicit hypotheses and conclusions.
\item Decompose the target into lemmas that can be proved from background facts.
\item Assemble lemmas to conclude the formal statement.

\end{enumerate}
\subsection{Key Lemmas}
\begin{itemize}
\item Lemma A: setup and normalization.
\item Lemma B: core transformation step.
\item Lemma C: final implication.

\end{itemize}
\subsection{Dependency Graph}
Lemma A -> Lemma B -> Lemma C -> Main theorem.

\subsection{Risky Steps}
\begin{itemize}
\item Items to watch from editor: None.
\end{itemize}

\subsection{Proof}
\subsection{Complete Proof}
We proceed by proving Lemmas A, B, and C, then conclude the theorem.
Each step uses only listed assumptions and previously proved lemmas.

\subsection{Lemma Proofs}
\begin{itemize}
\item Lemma A: established by direct unpacking of assumptions.
\item Lemma B: derived from Lemma A and background constraints.
\item Lemma C: follows from Lemma B and the target implication form.

\end{itemize}
\subsection{Gap Closure Notes}
Addressed editor feedback this loop: Provide a stepwise derivation for Lemma B. Break it into sub-claims and prove each from stated assumptions.

\subsection{Editor Dispatch}
\noindent\textit{Reasoning:} Demo: assigning all perspectives to the first pool reviewer.
\begin{itemize}
\item Correctness \& Completeness $\to$ default\_reviewer
\item Clarity \& Rigor $\to$ default\_reviewer
\item Reference Validity $\to$ default\_reviewer
\end{itemize}

\subsection{Reviewer Feedback}
\subsubsection{Correctness \& Completeness (by default\_reviewer)}
\noindent No issues identified.

\subsubsection{Clarity \& Rigor (by default\_reviewer)}
\noindent No issues identified.

\subsubsection{Reference Validity (by default\_reviewer)}
\noindent No issues identified.

\subsection{Editor Decision}
\noindent\textbf{Verdict:} \textcolor{green}{\textbf{accept}}

\noindent\textbf{Summary:} All reviewers are satisfied. The proof is correct and complete.

\section{Final Verdict}
\noindent The pipeline finished with verdict: \textcolor{green}{\textbf{accept}}.

\noindent The managing editor accepted the proof in loop 2.
\end{document}
